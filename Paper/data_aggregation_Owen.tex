\documentclass[twocolumn]{article}
\usepackage[utf8]{inputenc}
\usepackage[margin=0.8in]{geometry} %Margins
\usepackage{multicol} %Columns: multiples
\usepackage[hidelinks]{hyperref} %Hyperlinks
\usepackage{authblk} %Authors: affiliation
\renewcommand\Affilfont{\itshape\small} %Authors: affiliation font
\usepackage{tabularx} % Tabular: automatic line-break
\usepackage[justification=raggedright]{caption} %Floats: caption as with tables
\usepackage{ltablex} %Floats: long
\renewcommand{\arraystretch}{1.5} %Tables: more space between rows
\usepackage{graphicx} %Figures: import
\graphicspath{{../pics/}} %Figures path
\usepackage{subcaption} %Figures: multiple with one caption
\usepackage{amsmath} %Case definition
\usepackage[toc,page]{appendix} %Appendix
\newenvironment{conditions} %Conditions
{\par\vspace{\abovedisplayskip}\noindent\begin{tabular}{>{$}l<{$} @{${}={}$} l}} %Conditions
{\end{tabular}\par\vspace{\belowdisplayskip}} %Conditions
\usepackage{natbib} %Citations: APA
\usepackage{amsmath} %Use \[ ... \]
\usepackage{xcolor} %colors
\title{Food web aggregation: effects on key positions}
\author[1]{Emanuele Giacomuzzo}
\author[1]{Ferenc Jordàn}
\affil[1]{Stazione Zoologica Anton Dohrn, Napoli, 80122, Italy}
%\affil[2]{Central European University, Budapest, 1051, Hungary}
\date{}
\begin{document}
\maketitle
\section*{Introduction}

	Trophic data management is something that ecologists always must deal with when working with food webs. Trophic interactions can be described among individuals, life stages, species, higher taxa, functional groups, and several other, appropriately defined nodes of food webs. Some kind of aggregation is unavoidable, even the most highly resoluted food webs contain big aggregates (e.g., “bacteria'', see \citet{Martinez1991}). At the same time, even the least resoluted food webs may contain species (e.g., “hake”, see \citet{Yodzis1998}). Data aggregation can happen also at later stages, during data analysis, especially in large networks, where the study of hundreds of nodes would be unfeasible \citep{Yodzis1999}.

	Data aggregation methods are problem-dependent. Not considering this can bias the way by which we interpret the results of food web models \citep{Paine1988, Hall1993}. For instance, various levels of aggregation at different trophic levels might bias our interpretation if we are trying to characterise the structure of a network \citep{Yodzis1999}. Both low- and high-resolution networks can be useful or useless, the key challenge is to properly match the problem, the data management, and the model construction. Even if this seems like a ubiquitous problem in food web ecology, standards for whether and how to aggregate data meaningfully do not exist yet.

	The process of data aggregation assumes that there are nodes in the network that are similar enough that we can consider them functionally equivalent.
	For example, two fishes from the same genus might be aggregated into a node of the genus (e.g., \emph{Poecilia sphenops} and \emph{Poecilia reticulata} could be aggregated into \emph{Poecilia}).

	Similarity can be understood mathematically (equivalent network positions) and biologically (similar trophic habits).	\citet{Yodzis1999} and \citet{Luczkovich2003} tried to answer this question by borrowing two definitions from social networks. \citet{Yodzis1999} borrowed the concept of structural equivalence – where two nodes are similar when sharing a high number of neighbours – and called the aggregation of structurally equivalent species” trophospecies”.
	\citet{Luczkovich2003} borrowed the concept of regular equivalence – where two nodes are similar when sharing a high number of similar but not necessarily the same neighbours. Nodes belonging to the same equivalence class share ecological roles.

	Groups of nodes that have different neighbours but form dense subgraphs are called modules. Species in food web modules can play different roles (e.g. predator and prey) but they maintain well-defined multi-species processes (e.g. connecting benthic and pelagic organisms). Aggregating the modules of a food web has been suggested already by \citet{Allesina2009a}. The two most reliable ways of finding modules in food webs are through the group model and modularity maximisation. The group model was first developed by \citet{Allesina2009a} and then extended by \citet{Sander2015}.	Modularity maximisation was first applied to food webs by \citet{Guimera2010} following three definitions of modularity.	The first one, which we will refer to as density-based modularity, is the degree by which nodes inside modules interact more among themselves than with nodes of other modules.	The second one, which we will refer to as prey-based modularity, is the degree by which nodes inside modules interact with the same predators.	The third one, which we will refer to as predator-based modularity, is the degree by which nodes inside modules interact with the same preys.

	The positional importance of species differs in both highly aggregated and highly resoluted networks. Central positions may be a proxy for functional importance and the community-wide distribution of either centrality values \citep{Bauer2010} or hypothetical importance values \citep{Mills1993} provide macroscopic descriptors of ecosystems.

	In this paper, we investigate how these different aggregation methods maintain the relative importance of species, as a proxy of network structure.
	To compute the importance of species we used 25 of the most used centrality indices used in keystone species research.
	Our investigation was carried out on 86 Ecopath with Ecosim food web models. By having been constructed with the same method (see \citet{Okey2004}), they were easy to compare. These models were freely available on the EcoBase database \citep{Colleter2013}. The way we selected the food webs to be included in our analysis was the number of nodes: we selected only the food webs with at least 14 nodes. %CHANGE

						%\begin{figure}[htbp]%{\textwidth}
						%	\centering
						%	\includegraphics[width=1.0\linewidth]{reg_struct_equivalence}
						%	\caption{Two types of similarity indices highly used in social networks, which have been applied also to food webs. As you can see, two nodes are regularly equivalent if they are connected to similar nodes (b) and structurally equivalent if they are connected to the same exact nodes (a). Two structurally equivalent nodes are also regularly equivalent, but not the other way around. For example, two nurses are regularly equivalent because they have the same connections to other personnel in the hospital such as doctors, other nurses, receptionists, patients and so on. If the personnel they are in contact with not only is similar but it's the same exact persons, then they are also structurally equivalent.}
						%	\label{fig:equivalences}
						%\end{figure} %CHANGE

\section*{Methods: clustering techniques}

	To cluster similar nodes, we used the following clustering techniques.

	\subsection*{Hierarchical clustering with Jaccard index}

		As a first clustering method, we clustered structurally equivalent nodes as in \citet{Yodzis1999}, using the Jaccard similarity index as a measure of structural equivalence. See Appendix \ref{appendix:jaccard} for the clustering algorithm.

	\subsection*{Hierarchical clustering with REGE index}

		As a second clustering method, we clustered regularly equivalent nodes as in \citet{Luczkovich2003}, using REGE index as a measure of regular equivalence. See Appendix \ref{appendix:rege} for the clustering algorithm.

	\subsection*{Clustering of density-based modules}

		As a third clustering method, we clustered the nodes inside the modules found by maximising the density-modularity, as in \citet{Guimera2010}. This type of modularity is expressed as the number of extra links present within the modules compared to the ones expected by chance. For directed networks, it can be expressed through the following equation of \citet{Arenas2007}, which is a generalisation of the Newman-Girvan modularity \citep{Newman2004}:

						\begin{equation}
							Q=\frac{1}{L}\sum\limits_{ij}[A_{ij}-\frac{k_i^{in}k_j^{out}}{L}]\delta_{m_im_j} \label{eqn:modularitydensity}
						\end{equation}

		\noindent where $Q$ is the directed modularity of partition P, $L$ is the number of links in the network, $A_{ij}$ is the element of the adjacency matrix of a directed, binary network (links go from $j$ to $i$), $k_i^{in}k_j^{out}/L$ is the probability of having an edge between $i$ and $j$, $k^{in}_i$ is the indegree of $i$ and $k^{out}_j$ is the out-degree of $j$), $ m_i$ is the module of $i$, and $\delta$ is the Kronecker delta \citep{Kozen2007}.

		The number and composition of the modules were found by using the Leiden algorithm of \citet{Traag2019} - an extension of the Louvain algorithm \citep{Blondel2008}. The latter is one of the best performing and fastest for community detection \citep{Traag2019}. However, it tends to produce communities that are arbitrarily poorly connected from each other and sometimes even disconnected. The Leiden algorithm not only solves this problem by producing better-connected communities but it is also faster. The code that we used was implemented in the igraph package \citep{Csardi2006} for R \citep{RDevelopmentCoreTeam2011}.

	\subsection*{Clustering of prey-based and predator-based modules}

		As fourth and fifth clustering methods, we clustered the nodes of every module that was found by maximising the prey-modularity and the predator-modularity of the food web, as in \citet{Guimera2010}. Here, the modularity of the food web is expressed as to how much different nodes connect to the same predators (for prey-modularity) or preys (for predator-modularity) than expected by chance. Mathematically, it can be expressed by the following equation \citep{Guimera2007} for prey-modularity

						\begin{equation}
							Q=\sum_{ij}{\left[\frac{c_{ij}^{out}}{\sum_{l}{k_l^{in}\left(k_l^{in}-1\right)}}-\frac{k_i^{out}k_j^{out}}{\left(\sum_{l} k_l^{in}\right)^2}\right]\delta_{m_im_j}}
						\end{equation}

		or in the following one for predator-modularity

						\begin{equation}
							Q=\sum_{ij}{\left[\frac{c_{ij}^{in}}{\sum_{l}{k_l^{out}\left(k_l^{out}-1\right)}}-\frac{k_i^{in}k_j^{in}}{\left(\sum_{l} k_l^{out}\right)^2}\right]\delta_{m_im_j}}
						\end{equation}

		\noindent where $c_{ij}^{out}$ is the number of outgoing links that i and j have in common and $c_{ij}^{in}$ is the number of incoming links that i and j have in common.
		We maximised this type of modules by using the rnetcarto package \citep{Doulcier2015} for R. This finds the community structure of the network by using simulated annealing \citep{Kirkpatrick1983}.  %CHANGE

	\subsection*{Clustering of groups}

		As a sixth clustering method, we clustered the nodes inside the modules found by the group model of \citep{Allesina2009a}. This model finds the modules that maximise the probability of randomly retrieving the food web by generating a modular version of an Erdős-Rényi random graph. For an arbitrary number of groups $k$, the probability of retrieving the food web is:

				\begin{equation}
					P(N(S,L)|\vec{p}^{\,})=\prod_{i=1}^k\prod_{j=1}^k p_{ij}^{L_{ij}} (1-p_{ij})^{S_i S_j - L_{ij}}
				\end{equation}

		\noindent where $N(S,L)$ is the food web N with S number of nodes and L number of links,  $\vec{p}^{\,}$ is the vector containing the probabilities of a connection between and within clusters, $p_{ij}$ is the probability that a node inside the group $i$ connects to another node inside the group $j$, $L_{ij}$ is the number of links connecting nodes belonging to the group i to nodes belonging to the group j, $S_i$ is the number of nodes in the cluster i,  and $S_j$ is the number of nodes in the cluster j.

		Because of the high number of possible module arrangments, it is not possible to explore them all. To find the best possible solution that our computation power allows us to find, we used the algorithm of \citet{Sander2015}. This relies on a Metropolis-Coupled Markov Chain Monte Carlo ($MC^3$), also known as parallel tempering\citep{Geyer1991}, with a Gibbs sampler \citep{Yildirim2012}. $MC^3$ can be considered as a Markov chain Monte Carlo (MCMC) with multiple chains running all at once \citep{Sander2015}.

\section*{Methods: connecting the clusters and assigning interaction strength}

	The wiring of the food web followed a similar approach to the one describe in \citet{Martinez1991}. We used five methods to decide whether there was a link between two clusters. The first method produced the maximum connectance and is known as maximum linkage. Here, a cluster had a connection to another cluster if it had at least one link going from one of its nodes to the nodes of the second cluster. The second one produced the minimum connectance and is known as the minimum linkage. This time, a cluster was connected to another only if all its nodes had a connection to all the nodes of the other cluster. The other three methods produced an intermediate connectance. They considered a link from a cluster to the other only if at least 25\%, 50\% or 70\% of possible connections from the first cluster to the second were realised.

	The weight of the link was then calculated in four different ways: as the minimum weight, the maximum weight, the mean weight, and the sum of the weights of the links going from the members of the first cluster to the ones of the second cluster.

\section*{Methods: centrality indices}

	For each food web, we calculated the centrality indices before and after the aggregation. The centrality index of a node after the aggregation process equalled the one of its cluster. For example, let's consider a hypothetical aggregation. Before the aggregation, the node "hake" has a degree centrality of 5. Through the aggregation process, this happens to be aggregated into a fish cluster. The degree centrality of this fish cluster is 8. The degree centrality of hake was 5 before the aggregation and 8 after the aggregation.

	\subsection*{Degree centrality (DC)}

		The degree centrality ($DC$) of a node $i$ is the number of links it has \citep{Wasserman1994}

					\begin{equation}
								DC_i=\sum_{j=1}^{n}A_{ij}
					\end{equation}

		\noindent where $n$ is the number of nodes in the food web, and $A_{ij}$ is the element of the adjacency matrix after the network has been transformed in a binary undirected one. It can be normalised by dividing it by the total number of possible connections that a node could have \citep{Wasserman1994}

					\begin{equation}
								nDC_i=\frac{DC_i}{n-1}\ \ \
					\end{equation}

		\noindent where $n-1$ is the maximum number of connections the node can have. The minus one shows that a node cannot have a connection to itself.

		Another type of degree centrality that we considered was the weighted degree centrality ($wDC$), often referred to as node strength. Its formula, as well as the formula of its normalised version, are the same as for the non-weighted degree centrality. This time, however, the adjacency matrix is of an undirected weighted network \citep{Fornito2016}.

					\begin{equation}
								WDC_i=\sum_{j=1}^{n}A_{ij}
					\end{equation}

	\subsection*{Closeness centrality (CC)}

		The closeness centrality ($CC$) of a node is the average distance of a node from all the others \citep{Wasserman1994}

					\begin{equation}
								CC_i=\frac{1}{\sum\limits_{j=1}^n d(i,j)}
					\end{equation}

		\noindent where $d(i,j)$ is the shortest path between node $i$ and $j$. It can be normalised as follows \citep{Wasserman1994}

					\begin{equation}
								nCC_i=\frac{n-1}{\sum\limits_{j=1}^n d(i,j)}
					\end{equation}

	\subsection*{Betweenness centrality (BC)}

		The betweenness centrality ($BC$) of a node is the average number of times that it acts as a bridge along the shortest path between two other nodes. It can be  mathematically expressed as follows \citep{Wasserman1994}

						\begin{equation}
							BC_i=\sum_{i\neq m\neq n}\frac{\sigma_{mn}\left(i\right)}{\sigma_{mn}}
						\end{equation}

		\noindent where $\sigma_{mn}$ is the total number of shortest paths going from $s$ to $t$ and $\sigma_{mn}\left(i\right)$ is the total number of these paths passing through $i$. It can be normalised with the following equation \citep{Wasserman1994}

						\begin{equation}
							nBC_i=\frac{BC_i}{\left(n-1\right)\left(n-2\right)/2}
						\end{equation}

	\subsection*{Status index (s)}

		The status index of a node is the sum of its distances from all the other nodes inside the network, calculated as their shortest paths following a bottom-up direction \citep{Endredi2018}

						\begin{equation}
							s_i=\sum_{j=1}^{n}d\left(i,j\right).
						\end{equation}

		It was first introduced to social networks, followed two years later by its application to food webs by \citet{Harary1959, Harary1961}. By following the same method but in a top-down direction we obtain the controstatus $(s_i’)$

						\begin{equation}
							s_i^\prime=\sum_{j=1}^{n}d\left(i,j\right).
						\end{equation}

		The difference between the status and the controstatus is called the net status ($\Delta s_i$)

						\begin{equation}
							\Delta s_i=s_i-s_i^\prime.
						\end{equation}

	\subsection*{Keystone index (K)}

		The keystone index was first introduced by \citet{Jordan1999} and inspired by the status index. As the status index family, the keystone index of a species $i$ ($K(i)$) is calculated by considering the bottom-up and the top-down effects separately \citet{Jordan2006}

						\begin{equation}
							K\left(i\right)=K_b\left(i\right)+K_t\left(i\right)
						\end{equation}

		\noindent where $K_b\left(i\right)$ is its bottom-up keystone index of species $i$ and $K_t\left(i\right)$ the top-down keystone index of species $i$.

		Unlike the status index, which only considers the distance between a node and all the other nodes, the keystone index considers how the size of a certain effect gets split between the different neighbours of a node. Every time the effect reaches a certain node connected to multiple nodes, the following nodes receive only a fraction of the total effect. For example, when considering the bottom-up effect, if the prey has two predators, the bottom-up effect received by each predator will be half. The bottom-up effect of a certain node $(K_b\left(i\right))i$ is then calculated in the following way

						\begin{equation}
							K_b\left(i\right)=\sum_{j=1}^{n}\frac{1}{m\left(i\right)\left(j\right)}+\frac{K_b\left(j\right)}{m\left(i\right)\left(j\right)}
						\end{equation}

		\noindent where $j$ is a predator of $I$,$m(i)(j)$ is the number of preys of $j$, and $\frac{K_b\left(j\right)}{m\left(i\right)\left(j\right)}$ is the fraction of bottom-up effects of $j$ that are caused by $i$. The $K_b\left(j\right)$of top predators is set as 0. The top-down effect of a certain node $K_t\left(i\right)$ is calculated exactly as $K_b\left(i\right)$, but with the direction of the links inverted. The bottom-up and the top-down effects can also be split into their direct and indirect component. The indirect component considers the bottom-up effects of the predator and the direct component does not

					\begin{equation}
						K_{b,indirect}\left(i\right)=\sum_{j=1}^{n}\frac{1}{m\left(i\right)\left(j\right)}+\frac{K_b\left(j\right)}{m\left(i\right)\left(j\right)}
					\end{equation}

					\begin{equation}
						K_{b,direct}\left(i\right)=\sum_{j=1}^{n}\frac{1}{m\left(i\right)\left(j\right)}+\frac{1}{m\left(i\right)\left(j\right)}
					\end{equation}

		The direct and indirect components of the top-down effect are calculated in the same way, but with the direction of the links inverted. The direct and indirect keystone indices of a node are the sum of its direct/indirect bottom-up effects and its direct/indirect top-down effects

					\begin{equation}
						K_{direct}(i)=K_{b,direct}+K_{t,direct}
					\end{equation}

					\begin{equation}
						K_{indirect}(i)=K_{b,indirect}+K_{t,indirect}
					\end{equation}

		The keystone index not only is the sum of its top-down and bottom-up effects, but also the sum of its direct and indirect effects

					\begin{equation}
						K\left(i\right)=K_{dir}\left(i\right)+K_{indir}\left(i\right)
					\end{equation}

	\subsection*{Topological importance (TI)}

		The topological importance of a node represents its potential to create bottom-up effects on other species, up to a certain number of steps that we can set. It was first introduced to host-parasitoid networks by \citet{Muller1999} and then to food webs by \citet{Jordan2003}. The algorithm of its computation is reported in Appendix \ref{appendix:TI} \citep{Jordan2009}.



		Topological importance can be also used for weighted networks - giving us weighted topological importance ($WI$) – if instead of using the degree ($D$) we use the weighted degree ($WD$) \citep{Scotti2007}

						\begin{equation}
							a_{1,ji}=\frac{A_{ij}}{weighted\:indegree_j}
						\end{equation}

		\noindent where $A_{ij}$ is the element of the adjacency matrix of the weighted directed network.

	\subsection*{Trophic field overlap (TO) & species uniqueness (STO)}

		The trophic field overlap (TO) represents how redundant the strong interactions of a node are. It was first introduced by \citet{Jordan2009a}. It is the number of times that it and another node interact strongly with the same predator. The algorithm for its computation can be found in Appendix \ref{appendix:TO} \citep{Jordan2018}.


		Trophic field overlap can be also used for weighted networks – giving us weighted trophic field overlap (WO) – if instead of using the degree (D) we use the weighted degree, (e.g., \citet{Xiao2019})

						\begin{equation}
							a_{1,ij}=\frac{A_{ij}}{D_j}
						\end{equation}

		Finally, to avoid the bias of choosing a wrong threshold, we chose multiple thresholds and summed the TO of a species i for each of these thresholds. This gave us the species uniqueness (STO), an index that was first introduced by \citet{Lai2015}.

	\subsection*{Trophic position (TP)}

		The trophic position of a node is the mean length connecting it to the producers of the ecological community (its energy source). It was first introduced by \citet{Levine1980}, as a generalization of the earlier use of integer trophic levels to include fractional positions. It can be calculated through the following formula

						\begin{equation}
							TP_i=\sum\limits_{k=0}^\infty k \cdot p_i(k).
						\end{equation}

		\noindent where $k$ is a certain path length and $p_i(k)$ is the probability that species $i$ will reach the energy produced by the autotrophs via a path of length $k$. $TP$ equals 0 for producers, it equals 1 for herbivores and larger values for omnivores and carnivores.

\section*{Methods: statistical analysis}

	The combination of clusterings (6 methods), linkages (5 methods) and interaction strength determinations (4 methods) produced 120 aggregation methods. For each of these aggregation methods, we studied their effects on centrality indices. More in particular, for each centrality index, we studied how the nodes were ranked before and after the process. It was possible to study the difference between these two rankings by using Kendall's tau b ($\tau_B$) - a version of Kendall's rank correlation coefficient that makes adjustments for ties \citep{Agresti2012}. For each aggregation method and for each centrality index, we found the mean $\tau_B$ across all food webs. This required us to convert $\tau_B$ using the Fisher z-transformation \citep{Fisher1915}, find the mean and back-transform it. For each mean $\tau_B$ we found its confidence interval by bootstrapping \citep{DiCiccio1996}. $\tau_B$ and bootstrapping were implemented in the Statistics and Machine Learning Toolbox for MATLAB \citep{MathworksInc.2019}.

\section*{Results \& discussion}

		The 86 food webs had a median of 25.5 nodes (IQR = 16.0), with a minimum of 14 nodes and a maximum of 55 nodes. See Figure \ref{fig:food_web_sizes}. They produced a median number of 0.76 (IQR = 0.11) for the Jaccard index, 0.73 (IQR = 0.07) for the REGE index, 0.16 (IQR = 0.08) for the density modularity, 0.35 (IQR = 0.02) for the prey modularity, 0.16 (IQR = 0.08) for the predator modularity and 0.16 (IQR = 0.07) for the group model. See Figure \ref{fig:cluster_sizes}. There are some parts of my analysis that I would like to improve, so here I did not include the results and discussion. However, the main results will be as show in Figure \ref{fig:jaccard_results} - \ref{fig:groups_results}. These will include the mean Kendall's tau b coefficient showing how each aggregation method affected each of the centrality indices.

						\begin{figure}[htbp]%{\textwidth}
								\centering
								\includegraphics[width=1.0 \linewidth]{food_web_sizes.jpg}
								\caption{Size of the 86 food webs used in this study.}
								\label{fig:food_web_sizes}
						\end{figure}

						\begin{figure}[htbp]%{\textwidth}
								\centering
								\includegraphics[width=1.0\linewidth]{cluster_sizes.jpg}
								\caption{Number of clusters produced by the different aggregation methods. (a) = hierarchical clustering with Jaccard index, (b) = hierarchical clustering with REGE index, (c) = density-based modules, (d) = prey-based modules, (e) = predator-based modules, (f) = groups.}
								\label{fig:cluster_sizes}
							\end{figure}

\section*{Acknowledgements}

	We would like to thank Wei-Chung Liu for providing the code for computing some centrality indices and Stefano Allesina \& Elizabeth Sander for providing the code for the computation of the group model. Ferenc Jordàn was supported by H2020 AtlantECO.

\section*{Supplementary material}

	The code of the analysis will be available at \url{https://github.com/Emanuele-Giacomuzzo/Data_aggregation}.

\bibliographystyle{apalike}
\bibliography{/Applications/Mendeley/library.bib}

\onecolumn
\begin{appendices}

	\section{Hierarchical clustering with Jaccard index} \label{appendix:jaccard}

		\begin{enumerate}

			\item \emph{Compute similarity.} \smallskip \newline
						Compute the Jaccard similarity between the nodes by using the following equation \citep{Yodzis1999}:

										\begin{equation}
				      				J_{ij}=\frac{a}{a+b+c} \label{eqn:jaccard}
			      				\end{equation}

			      \noindent where $J_{ij}$ is the Jaccard similarity between node $i$ and $j$, $a$ is the number of preys and predators that $i$ and $j$ have in common, $b$ is the number of preys and predators only of i, and $c$ is the number of preys and predators only of $j$.

			\item \emph{Build the dendrogram.} \smallskip \newline
			      Find the two most similar elements and cluster them together (elements are intended as nodes or clusters. Of course, the first time we run this step all the elements are nodes). Repeat until you are left with only one item, which is the final dendrogram. During this process, the similarity between two clusters can be calculated in different ways, called linkage criteria. The ones that we used were

						\begin{itemize}
				      \item 	The similarity between the least similar nodes, one in each cluster, known as single-linkage \citep{Frigui2008}.
				      \item 	The similarity between the most similar nodes, one in each cluster, known as complete linkage \citep{Frigui2008}.
				      \item 	The mean similarity between the nodes inside the first item and the second item, known as the weighted average distance(WPGMA) \citep{Sokal1958}:

				            	\begin{equation}
					            	d_{(i \bigcup j),k}=\frac{d_{i,k}+d_{j,k}}{2} \label{eqn:WPGMA}
				            	\end{equation}

											\noindent where $d_{\left(i\cup j\right),k}$ is the distance between the cluster $i \bigcup j$ (cluster including $i$ and $j$) and $k$, $d_{i,k}$ is the distance between $i$ and $k$, and  $d_{j,k}$ is the distance between $j$ and $k$.
				      \item	The mean similarity between the nodes inside the first item and the second item, but taking into consideration the average distance between the items inside the fist cluster; this is known as the unweighted average distance (UPGMA) \citep{Sokal1958}:

				            	\begin{equation}
					            	d_{(i \bigcup j),k}=\frac{|i|d_{i,k}+|j|d_{j,k}}{|i|+|j|} \label{eqn:UPGMA}
				            	\end{equation}

							\noindent where $|i|$ and $|j|$ are the mean distances between the elements inside $i$ and $j$, respectively.
			      \end{itemize}

			\item \emph{Select the dendrogram.} \smallskip \newline
						After having produced a dendrogram for every linkage criteria, select the dendrogram with the highest cophenetic correlation \citep{Sokal1962}.
			      This allows selecting the linkage criterion that produces the dendrogram that preserves the most faithfully the pairwise similarity between different elements.

			\item \emph{Cut the dendrogram.} \smallskip \newline
			      Cut the dendrogram according to the maximum inconsistency of the branches, set at 0.01.

		\end{enumerate}

	\section{Hierarchical clustering with REGE index} \label{appendix:rege}

		\begin{enumerate}

			\item \emph{Compute similarity.} \smallskip \newline
			Compute the similarity between nodes by using REGE, calculated by the homonym algorithm. This was originally developed in the unpublished work by \citet{White1980,White1982,White1984} and first described in the literature by \citet{Borgatti1993}. It is available to be used in the software UCINET VI \citet{Borgatti2002}. The REGE algorithm is as follows \citep{Jordan2018}:

				      \begin{enumerate}

					      \item Set the maximum number of iterations. We set 3 iterations. Each iteration produces a matrix $R_{\left(t\right)}$ where $t$ is the number of the iteration and every element $r_{\left(t\right)ij}$ is the regular equivalence between i and j at iteration t. The regular equivalence between nodes at iteration t=0 is always 1.

					      \item Starting from t=1, update the elements of the matrix following these sub-steps:

					            \begin{enumerate}
						            \item For every predator k of species i, find the most similar predator m of species j according to  $R_{\left(t\right)}$.
						                  Now, set $X_{i,k,j}=R_{\left(t\right)km}.$
						            \item For every predator m of species j, find the most similar predator k of species o according to $R_{\left(t\right)}$.
						                  Now, set $X_{j,m,i}=R_{\left(t\right)mk}$.
						            \item For every prey h of species i, find the most similar prey n of species j according to $R_{\left(t\right)}$.
						                  Now, set $Y_{i,h,j}=R_{\left(t\right)hn}$.
						            \item For every prey n of species j, find the most similar prey h of species i according to $R_{\left(t\right)}$.
						                  Now, set $Y_{j,n,i}=R_{\left(t\right)nh}$.
						            \item Update the matrix R through the following equation
						                  %\begin{equation}
						                  % R_{\left(t\right)ij}=\frac{\sum_{k=1} X_{i,k,j}+\sum_{m=1} X_{j,m,i}+\sum_{h=1} Y_{i,h,j}+\sum_{n=1} Y_{j,n,i}}
						                  %{MAX\left(\sum_{k=1} X_{i,k,j}+\sum_{m=1} X_{j,m,i}+\sum_{h=1} Y_{i,h,j}+\sum_{n=1} Y_{j,n,i}\right)}
						                  %\end{equation}
						            \item Increase t=t+1 and repeat step b until you reach the maximum number of iterations. The matrix of the maximum number of iterations contains the regular equivalence between nodes.
					            \end{enumerate}

					      \item Increase t=t+1 and repeat step b until you reach the maximum number of iterations. The matrix of the maximum number of iterations contains the regular equivalence between nodes.

				      \end{enumerate}

			\item \emph{Build the dendrogram.} \smallskip \
			The same as in the hierarchical clustering of nodes according to their Jaccard similarity index. During our analysis, we used the function linkage of MATLAB, which does not include the possibility of using a similarity matrix, so we converted the similarity matrices into dissimilarity ones. This was done by following what was written in \citet{VonLuxburg2004}. Namely, if the similarity function is normalised - takes values between 0 and 1 - and always positive, then $d=1-s$ where d is the dissimilarity measure and s is the similarity measure).

			\item \emph{Select the dendrogram.} \smallskip \newline
			The same as in the hierarchical clustering of nodes according to their Jaccard similarity index.

			\item \emph{Cut the dendrogram.} \smallskip \newline
			The same as in the hierarchical clustering of nodes according to their Jaccard similarity index.

		\end{enumerate}

	\section{Topological importance computation} \label{appendix:TI}

		\begin{enumerate}

			\item \emph{Compute the one-step matrix.} \smallskip \newline
			In the one-step matrix, if the energy flows from a prey to the predator, then the effect of the prey on the predator is the reciprocal of the indegree of the predator

								\begin{equation}
									a_{1,ij}=\frac{A_{ij}}{D_j}
								\end{equation}

			\item \emph{Compute the n-step matrices.} \smallskip \newline
			In the higher steps matrices, a node influences another node at a higher trophic level by summing the effects of every path that connects the two nodes. The effect of every path is the multiplication of the inverse of the outdegree of every node along the path. For a visual explanation see Figure \ref{fig:TI}. It can be calculated as follows

								\begin{equation}
									A\left(n\right)=A_{\left(1\right)}^n
								\end{equation}

			\item \emph{Calculate topological importance} \smallskip \newline
			The topological importance of a node i ($TI_i$) can be calculated through the following formula

								\begin{equation}
									TI_i=\frac{\sum\limits^N_{m=1}\sum\limits^n_{j=1}a_{m,ji}}{N}
								\end{equation}

			\noindent where $N$ is the total number of steps considered, $m$ is the step number,  $n$ is the total number of nodes, and $a_{m,ji}$ is the effect of species $i$ on species $j$ at $m$ number of steps.

								\begin{figure}[htbp]%{\textwidth}
									\centering
									\includegraphics[width=1.0\linewidth]{TI_example.png}
									\caption{Topological importance (TI) of a species on another. The plant and the bear are not connected, so there is no direct effect from the plant to the bear. However, indirect effects reach the bear from the plant through two paths and the final effect is the sum of these effects. The first one is through the deer and the second is through the fox. The strength of these paths is the product of the direct effects composing the path. The first path has an effect on the bear that is 0.9*0.6=0.54, the second one has an effect on the bear that is 0.1*0.4=0.04. Summing the effects through these two 2-step paths connecting the plant with the bear, we get the 2-step effect of the plant on the bear: 0.54+0.04=0.58. Figure created with BioRender.com.} \label{fig:TI}
								\end{figure}

		\end{enumerate}

	\section{Trophic field overlap computation} \label{appendix:TO}

		\begin{enumerate}
			\item Compute the one-step matrix as in topological importance \item Compute the n-step matrix as in topological importance \item Compute the average effect matrix. The average effect matrix ($E(n)$) represents the effect of each node on the other nodes average by the number of steps

					\begin{equation}
						E_n=\frac{1}{n}\sum_{i=1}^{n}A_{\left(i\right)}
					\end{equation}

			\item Compute the interactor matrix.
					Compute the so-called interactor matrix ($M_T$), whose values tell us whether the interaction between two nodes is weak (W) or strong (S).
					To do this, we need to define a threshold over which a certain interaction is strong.
		  \item Compute the topological overlap matrix.
					Compute a matrix with how many times two species interact strongly with the same predator, called the topological overlap matrix.
		  \item Compute the trophic field overlap (TO)
					The trophic field overlap (TO) of a node is calculated by summing the elements of the rows of the topological overlap matrix.
		\end{enumerate}

	\section{Figure of future results}

		\begin{figure}[b]%{\textwidth}
			\centering
			\includegraphics[width=1.0 \textwidth]{heatmap_jaccard.jpg}
			\caption{Heat map of Kendall's tau b showing how different aggregations change the rankings according to centrality indices. It is not available yet. } \label{fig:jaccard_results}
		\end{figure}

		\begin{figure}[b]%{\textwidth}
			\centering
			\includegraphics[width=1.0 \textwidth]{heatmap_rege.jpg}
			\caption{Heat map of Kendall's tau b showing how different aggregations change the rankings according to centrality indices. It is not available yet. } \label{fig:rege_results}
		\end{figure}

		\begin{figure}[b]%{\textwidth}
			\centering
			\includegraphics[width=1.0 \textwidth]{heatmap_density.jpg}
			\caption{Heat map of Kendall's tau b showing how different aggregations change the rankings according to centrality indices. It is not available yet. } \label{fig:density_results}
		\end{figure}

		\begin{figure}[b]%{\textwidth}
			\centering
			\includegraphics[width=1.0 \textwidth]{heatmap_predators.jpg}
			\caption{Heat map of Kendall's tau b showing how different aggregations change the rankings according to centrality indices. It is not available yet. } \label{fig:predators_results}
		\end{figure}

		\begin{figure}[b]%{\textwidth}
			\centering
			\includegraphics[width=1.0 \textwidth]{heatmap_preys.jpg}
			\caption{Heat map of Kendall's tau b showing how different aggregations change the rankings according to centrality indices. It is not available yet. } \label{fig:preys_results}
		\end{figure}

		\begin{figure}[b]%{\textwidth}
			\centering
			\includegraphics[width=1.0 \textwidth]{heatmap_groups.jpg}
			\caption{Heat map of Kendall's tau b showing how different aggregations change the rankings according to centrality indices. It is not available yet. } \label{fig:groups_results}
		\end{figure}

\end{appendices}
\end{document}
