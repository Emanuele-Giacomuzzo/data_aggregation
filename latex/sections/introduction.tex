\section*{Introduction}

	Trophic data management is something that ecologists always must deal with when working with food webs. Trophic interactions can be described among individuals, life stages, species, higher taxa, functional groups, and several other, appropriately defined nodes of food webs. Some kind of aggregation is unavoidable, even the most highly resoluted food webs contain big aggregates (e.g., “bacteria'', see \citet{Martinez1991}). At the same time, even the least resoluted food webs may contain species (e.g., “hake”, see \citet{Yodzis1998}). Data aggregation can happen also at later stages, during data analysis, especially in large networks, where the study of hundreds of nodes would be unfeasible \citep{Yodzis1999}.

	Data aggregation methods are problem-dependent. Not considering this can bias the way by which we interpret the results of food web models \citep{Paine1988, Hall1993}.For instance, various levels of aggregation at different trophic levels might bias our interpretation if we are trying to characterise the structure of a network \citep{Yodzis1999}.Both low- and high-resolution networks can be useful or useless, the key challenge is to properly match the problem, the data management, and the model construction. Even if this seems like a ubiquitous problem in food web ecology, standards for whether and how to aggregate data in a meaningful way does not exist yet.

	The process of data aggregation assumes that there are nodes in the network that are similar enough that we can consider them functionally equivalent.
	For example, two fishes from the same genus might be aggregated into a node of the genus (e.g., \emph{Poecilia sphenops} and \emph{Poecilia reticulata} could be aggregated into \emph{Poecilia}).

	Similarity can be understood mathematically (equivalent network positions) and biologically (similar trophic habits).	\citet{Yodzis1999} and \citet{Luczkovich2003} tried to answer this question by borrowing two definitions from social networks. \citet{Yodzis1999} borrowed the concept of structural equivalence – where two nodes are similar when sharing a high number of neighbours – and called the aggregation of structurally equivalent species” trophospecies”.
	\citet{Luczkovich2003} borrowed the concept of regular equivalence – where two nodes are similar when sharing a high number of similar but not necessarily the same neighbours. Nodes belonging to the same equivalence class share ecological roles.

	Groups of nodes that have different neighbours but form dense subgraphs are called modules. Species in food web modules can play different roles (e.g. predator and prey) but they maintain well-defined multi-species processes (e.g. connecting benthic and pelagic organisms). Aggregating the modules of a food web has been suggested already by \citet{Allesina2009a}. The two most reliable ways of finding modules in food webs are through the group model and modularity maximisation. The group model was firstly developed by \citet{Allesina2009a} and then extended by \citet{Sander2015}.	Modularity maximisation was firstly applied to food webs by \citet{Guimera2010} following three definitions of modularity.	The first one, which we will refer to as density-based modularity, is the degree by which nodes inside modules interact more among themselves than with nodes of other modules.	The second one, which we will refer to as prey-based modularity, is the degree by which nodes inside modules tend to interact with the same predators.	The third one, which we will refer to as predator-based modularity, is the degree by which nodes inside modules tend to interact with the same preys.

	The positional importance of species differs in both highly-aggregated and highly-resoluted networks. Central positions may be a proxy for functional importance and the community-wide distribution of either centrality values \citep{Bauer2010} or hypothetical importance values \citep{Mills1993} provide macroscopic descriptors of ecosystems.

	In this paper, we investigate how these different aggregation methods maintain the relative importance of species, as a proxy of network structure.
	To compute the importance of species we used 25 of the most used centrality indices used in keystone species research.
	Our investigation was carried out on 86 Ecopath with Ecosim food web models. By having been constructed with the same methodology (see \citet{Okey2004}), they were easy to compare. These models were freely available on the EcoBase database \citep{Colleter2013}. The way we selected the food webs to be included in our analysis was the number of nodes: we selected only the food webs with at least 14 nodes.  See a table of these food webs in Tables S1 from \citet{Heymans2014}. %CHANGE
